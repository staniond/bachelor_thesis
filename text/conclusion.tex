%---------------------------------------------------------------
\chapter*{Conclusion}\label{sec:conclusion}
\phantomsection
\addcontentsline{toc}{chapter}{Conclusion}
%---------------------------------------------------------------

In this thesis, the possibility of implementing a \gls{sram} \gls{puf} on the ESP32 microcontroller was analyzed and a proof of concept solution was implemented. 

First, a description of the concept of a \gls{puf} was provided. \gls{puf} properties, applications, classification and evaluation parameters were described and three \gls{puf} implementations were presented.

A thorough explanation of \gls{sram} \glspl{puf} was given. A discussion on which properties the \gls{sram} \gls{puf} possesses was provided. Then, stable bit preselection methods and the effect of silicon aging were presented and existing research on \gls{sram} \glspl{puf} on the ESP32 platform was summarized. An overview of the ESP32 platform was also provided. The presented theory was used in the \gls{sram} \gls{puf} experiments and implementation in this thesis.

Then, two power-control methods of the \gls{sram} memory were presented, the \gls{rtc} \gls{sram} method and the deep sleep method. To assess the performance of the \gls{sram} \gls{puf} that will use these methods, several experiments were conducted. Data remanence of the \gls{sram} memory and evaluation parameters were analyzed across a range of temperatures. The \gls{rtc} \gls{sram} method showed a surprising effect of memory `freezing' and was found to be suitable for use only at temperatures above about 10 °C. The deep sleep method showed very good results according to the evaluation parameters and data remanence testing. However, it is significantly slower compared to the former method. Additionally, \gls{puf} response visualization revealed an interesting pattern of the uninitialized \gls{sram} data in some \glspl{mcu}, which reduces the entropy of the \gls{puf} responses.

Based on the analyses conducted, a \gls{sram} \gls{puf} implementation with reliable response extraction was designed. Two methods of stable bit preselection were implemented and tested. Then, a simple repetition \gls{ecc} was used to further stabilize the responses. A simple mathematical model of reconstruction success was used to evaluate the reliability of the \gls{puf}. Next, a \gls{puf} design that combines the two power-control methods was presented. By default, the fast \gls{rtc} \gls{sram} method is used, with the slower deep sleep method used as a fallback mechanism. This design was successfully tested on 16 \glspl{mcu} across a range of temperatures.

Finally, the described \gls{sram} \gls{puf} design was implemented in an easy-to-use ESP32 library together with an in-device and external enrollment. The \gls{puf} responses can be used to generate and preserve cryptographic keys safely, increasing the security of the ESP32 platform.

To summarize, all of the goals outlined were accomplished, resulting in interesting findings and a working \gls{sram} \gls{puf} implementation on the ESP32 microcontroller.

\pagebreak

\section*{Future work}
\phantomsection
\addcontentsline{toc}{section}{Future work}

At the end, ideas for possible future research are presented:
\begin{itemize}
    \item test the \gls{puf} responses using more evaluation parameters (such as randomness or bit-aliasing)
    \item use more advanced \glspl{ecc} to achieve greater reliability without the need for sophisticated stable bit preselection
    \item evaluate the behaviour of the \gls{puf} depending on other operating conditions (for example supply voltage)
    \item test the \gls{sram} \gls{puf} implementation on the new RISC-V based ESP32 microcontroller models
    \item apply the reconstructed key to secure communication on ESP32
\end{itemize}

% pro DPR staci jen nastrel
% zopakovat cile, napsat jestli a jak jsem je splnil a future work
