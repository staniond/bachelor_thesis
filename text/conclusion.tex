%---------------------------------------------------------------
\chapter*{Conclusion}\label{sec:conclusion}
\phantomsection
\addcontentsline{toc}{chapter}{Conclusion}
%---------------------------------------------------------------

In this thesis, the possibility of implementing a \gls{sram} \gls{puf} on the ESP32 microcontroller was analyzed and a proof of concept solution was implemented. 

The first chapter provides a description of the concept of a \gls{puf}. Then, \gls{puf} properties, applications, classification and evaluation parameters were described. At the end, three \gls{puf} implementations were presented.

A thorough explanation of \gls{sram} \glspl{puf} was given in the second chapter. A discussion on which properties the \gls{sram} \gls{puf} possesses was provided. Then, stable bit preselection methods and the effect of silicon aging were presented. At the end, existing research on \gls{sram} \glspl{puf} on the ESP32 platform was summarized.

In the third chapter, an overview of the ESP32 \gls{iot} platform was given. Hardware and software components of the ESP32 microcontroller with a focus on features specific to the \gls{sram} \gls{puf} implementation were introduced.

Two \gls{sram} power state control methods on the ESP32 microcontroller were designed in the fourth chapter. Their resulting \gls{puf} responses were analyzed based on operating temperature and power-off time of the memory. This analysis detected an interesting effect of memory `freezing' for the \gls{rtc} \gls{sram} method in low operating temperatures. Then, evaluation parameters were used to test the \gls{puf} responses and the results matched the expectations from the previous analysis. The advantages and disadvantages of each method and their suitability for use were also discussed. At the end, a visualization of the \gls{puf} responses was presented, revealing an interesting pattern in the memory startup values.

Chapter five implemented reliable \gls{puf} response reconstruction using two methods of stable bit preselection and a simple repetition \gls{ecc}. Next, a \gls{puf} design that combines the two power-control methods was presented. At the end, the proposed \gls{sram} \gls{puf} design was tested on 16 \glspl{mcu} in different operating temperatures.

The final chapter introduced esp32\_puflib---a library that implements the proposed \gls{sram} \gls{puf} design.

To summarize, all of the goals outlined were accomplished, resulting in interesting findings and a working \gls{sram} \gls{puf} implementation on the ESP32 microcontroller.

\newpage

\section*{Future work}
\phantomsection
\addcontentsline{toc}{section}{Future work}

At the end, ideas for possible future research are presented:
\begin{itemize}
    \item test the \gls{puf} responses using more evaluation parameters (such as randomness or bit-aliasing)
    \item use more advanced \glspl{ecc} to achieve greater reliability without the need of sophisticated stable bit preselection
    \item evaluate the behaviour of the \gls{puf} depending on other operating conditions (for example supply voltage)
    \item test the \gls{sram} \gls{puf} implementation on the new RISC-V based ESP32 microcontroller models
    \item apply the reconstructed key to secure communication on ESP32
\end{itemize}

% pro DPR staci jen nastrel
% zopakovat cile, napsat jestli a jak jsem je splnil a future work
