%---------------------------------------------------------------
\chapter{Introduction}
%---------------------------------------------------------------

    This is an introduction to my thesis\ldots

%--------------------------------
\section{Objectives of the thesis}
%--------------------------------

The research part of the thesis will describe what is a \gls{puf} and what are the uses of this technology, focusing mainly on \gls{sram} based \glspl{puf}. Furthermore, the parameters used to evaluate performance of \glspl{puf} will be defined.

The main objective of the practical part of this thesis is to analyze the possibility of implementing a \gls{sram} based \gls{puf} on the ESP32 family of microcontrollers. This means finding a suitable mechanism of \gls{sram} power control and establishing a way to gather \gls{puf} response data reliably. Next, stability and uniqueness of the obtained \gls{puf} data using the defined parameters will be evaluated. An analysis of the \gls{sram} data depending on different operating temperature and power-off time will be performed as well.

The next goal will be to implement a simple \gls{sram} based \gls{puf} based on the knowledge gained in the previous experiments. Several different models of devices from the ESP32 family will be used to test the function and performance of the resulting \gls{puf} implementation.

\section{Structure of the thesis}

%---------------------------------------------------------------
\chapter{Physical unclonable functions}
%---------------------------------------------------------------

%--------------------------------
\section{PUF description}
%--------------------------------

As \glspl{puf} are the main subject of this thesis, it is important to provide a thorough explanation of what a \gls{puf} is, what types and classes exist and what are
the applications of this technology.

Since more and more types of \glspl{puf} are being invented, it turns out that creating a generalizable description is not a straightforward task. A dictionary definition of a \gls{puf} could be expressed as: \say{a PUF is an expression of an inherent and unclonable instance-specific feature of a physical object}. One can imagine a \gls{puf} being an object's fingerprint in a very similar way to how humans have their own fingerprints.\cite{Maes2013}

The first concept of \gls{puf} was proposed by Pappu in 2001. He used the term \gls{powf}, which he described as a function operating on a physical system which could be easily computed but not easily inverted.\cite{Pappu2001} The first mention of the term \gls{puf} was by Gassend et al in 2002. He talks about \glspl{prf} and a \gls{puf} implementation using \glspl{fpga}.\cite{Gassend2002}

As \gls{puf} is a function, it has inputs and outputs. However, it is not a function in a true mathematical way. It could be described as a procedure performed on a particular device. Its inputs consist of a challenge and a physical state of the device. Given the input, the \gls{puf} produces an output (called a response). Together, they form challenge-response pairs.   

A very important property of \glspl{puf} is unclonability. It is achieved by the physical state of the device which acts as the input to the function and influences the responses produced by the \gls{puf}. The concrete details of the physical state used is what distinguishes different \gls{puf} implementations. These physical properties could be for example propagation delay in the chip circuit or bias of uninitialized memory cells to 1 or 0 state. The latter is a basis for \gls{sram} \gls{puf}, which is a topic of this thesis. These properties are fundamentally random, since they are created by uncontrollable physical processes during manufacturing. This makes them physically unclonable.

Since the physical state of the device can change with time and environment (for example temperature or input voltage variations), the challenge-response pairs can change as well. The requirement is, that for the same challenge, the responses should be similar enough for us to be able to recognize that they belong to the same challenge. The \gls{puf} responses are also required to differ from device to device even with the same challenge.\cite{Kodytek2020}

Because of these properties, \glspl{puf} can be used in devices to enable secure identification, authentication as well as for cryptographic key generation. More of the required properties of \glspl{puf}, their classification, possible applications and implementations will be discussed in more detail in the next sections.

%--------------------------------
\section{PUF properties}
%--------------------------------


%--------------------------------
\section{PUF classification}
%--------------------------------

%--------------------------------
\section{PUF evaluation parameters}
%--------------------------------

\subsection{Uniformity}
\subsection{Uniqueness}
\subsection{Reliability}
\subsection{?Randomness}

%--------------------------------
\section{PUF applications}
%--------------------------------

\subsection{Identification}
\subsection{Authentication}
\subsection{Key generation}

%--------------------------------
\section{PUF implementations}
%--------------------------------

\subsection{?other PUFs}
\subsection{SRAM PUF}

%---------------------------------------------------------------
\chapter{ESP32 platform}
%---------------------------------------------------------------

%---------------------------------------------------------------
\chapter{SRAM PUF implementation on ESP32}
%---------------------------------------------------------------

%--------------------------------
\section{RTC SRAM based PUF}
%--------------------------------

\subsection{RTC SRAM power control}
\subsection{SRAM analysis based on temperature and power off time}
\subsection{PUF evaluation parameters}

%--------------------------------
\section{Deep sleep based PUF}
%--------------------------------

\subsection{Deep sleep SRAM power control}
\subsection{SRAM analysis based on temperature and power off time}
\subsection{PUF evaluation parameters}

%--------------------------------
\section{?PUF response data image} % TODO how to name this section?
%--------------------------------

%---------------------------------------------------------------
\chapter{Reliable PUF response extraction} % TODO should this be its own section or part of the previous one?
%---------------------------------------------------------------

%--------------------------------
\section{Stable bits selection}
%--------------------------------

%--------------------------------
\section{Error correction code}
%--------------------------------

%--------------------------------
\section{Provisioning}
%--------------------------------

%--------------------------------
\section{Combining power control methods}
%--------------------------------

%--------------------------------
\section{Reliability testing}
%--------------------------------

%---------------------------------------------------------------
\chapter{ESP32 SRAM PUF library}
%---------------------------------------------------------------

%---------------------------------------------------------------
\chapter{Conclusion}
%---------------------------------------------------------------

